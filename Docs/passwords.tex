\subsection{Windows}

Windows se chovají jinak v doméně a jinak mimo ní. Pokud je počítač v doméně je preferován autentizační protokol kerberos. V současných Windows Server edicích je implementován Kerberos verze 5. Kerberos v základní nastavenim operuje na portu 88 a k šifrování používá symetrickou šifru. 
Pokud počítač není nastaven aby se autentikoval pomocí protokolu Kerberos používají Windows šifrování NTLM.

\subsection{Linux}

Hesla v linuxových systémech se skládají ze dvou konkretních souborů. 

/etc/shadow - obsah a strukturu toho souboru můžeme vidět na následujícím obrázku. 

% TODO Pictore of the structure  

/etc/passwd - obsah a strukturu tohoto souboru můžeme vidět na následujícím obrázku.

% TODO Pictore of the structure 

V /etc/shadow jsou hesla uložená pomocí hashe. 

\subsection{MacOS}

%TODO MacOS password handling

\section{Útoky na hesla}

\subsection{Hrubou silou}

\subsection{Pomocí masky}

\subsection{Se slovníkem}

\section{Entropie hesla}

\section{Ochrana před různými útoky}

