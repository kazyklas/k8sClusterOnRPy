\section{Kubernetes}

Jméno Kubernetes pochází z Řecka a znamená to kormidelník. Projekt zložili Joe Beda, Brendan Burns, a Craig McLuckie,ke kterým se rychle připojili inženýři z Googlu, jako Brian Grant a Tim Hockin. Software byl vydán v roce 2014.

Kubernetes je opensource technologie pro vytvoření a správu klastru. Pomáhá na tomto klastru plánovat spuštění kontejnerů na základě jeho stavu. Řídi automatickou aktualizaci kontejnerů a jejich opravu. Kontejnery sdružuje do podů, což je základní jednotka pro Kubernetes. Tyto pody škáluje na požadovaný stav.
Kubernetes také vyvažují zatížení a v případě pádu aplikace restartuje kontejner, aby znovu splnily požadavky.

\subsection{Stavební kameny Kubernetes}

Aby Kubernetes zajistili funkčnost klastru je zapotřebí rozdělit práci do několika komponent, které se starají o správný chod. Níže je znázorněno, z čeho se skládají. Dále se komponenty popíší a vysvětlí se na nich kompletní funkcionalita. 

\subsubsection{Master}



\subsubsection{Pod}



\subsubsection{Plánovač}



\subsubsection{Controller Manager}



\subsubsection{API server}



\subsubsection{Kubelet}



\subsubsection{Kube-Proxy}



\subsection{Kubernetes a tajemství}



\subsection{Aktualizace obraz}



\subsection{Sdílené datové prostory}



\subsection{Cloudový poskytovatelé}



\subsection{Flannel}

