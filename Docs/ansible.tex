\section{Ansible}

Ansible je automatizační nástroj pro konfiguraci systému, nasazení softwaru, aktualizac. Jeho nejsilnější stránka je nulové výpadky systému při aktualizaci balíčků, nebo automatické nastavovat dané zařízení. 

Jeho hlavními cíly jsou jednoduchost a nenáročnost. Kód by měl být čitelný i pro lidi, kteří nejsou obeznámeni s programem. Je schopen pokrýt různě velké prostředí od malých podniků až po velice obsáhlou infrastrukturu. 

Ansible se připojí na vzdálený počítač pomocí OpenSSH pomocí uživatele, který je současně přihlášen. Na spravovaném počítači není trřeba žádný agent. Je možnost nakonfigurovat Ansible, aby pro připojení nepoužíval OpenSSH, ale i kerberos nebo LDAP. 

Ansible a jemu podobné nástroje se použijí v případě toho, že máme více serverů, nebo pokud budeme dodržovat trend IaaS. Kdyby měl admin ve firmě nasazovat nebo aktualizovat například sto počítačů, tak na každém stráví 15 minut. To bude 1500 minut a to je 25 hodin. Proto použije Ansible a za 20 minut je hotov. 

Pro příklad budeme instalovat Docker na 3 počítače. Na těchto počítačích budeme mit v souboru /root/.ssh/authorized_keys naše veřejné klíče pro ssh. Na hostovi, ze kterého budeme instalovat je potreba Ansible. Tedy "sudo apt install ansible". V souboru hosts, který je pro Ansible tzv. inventářem naspecifikujeme stanice, na které budeme instalovat Docker. 

IP ansible_user=root
IP ansible_user=root
IP ansible_user=root

V souboru deployment.yml se specifikuje, co se bude spouštět a na jakých stanicích. 

= hosts: all
..
..


Ve složce tasks dale napíšeme script, který se spustí pomocí pythonu a jeho modulů na vzdálené stanici.

- name 
  apt"
  	instal: docker.io
  	
  	
  	

\subsection{Komponenty}

\subsubsection{Control node}

Jakýkoliv počítač s nainstalovaným Ansible a pythonem, může spouštět příkazy nebo tzv. playbooky. Tento počítač se nazývá control node. Takových můžeme mít klidně více, ne však počítače, které mají nainstalovaný operační systém Windows. 

\subsubsection{Managed node}

Je jakékoliv síťové zařízení. Managed nodes můžeme také nayývat jako hosts. Tyto zařízení nemusejí mít nainstalovaný Ansible, ale musejí mít nainstalovaný python. Ansible může být nakofigurován, aby používal specifikovanou verzi pythonu, pokud není specifikována, spustí se na hostu jeho defaultní.

\subsubsection{Iventory}

Je seznam všech nastavovaných zařízení. Často se nazývá hostfile. v tomto souboru nastavujeme skupiny zařízení, jejich IP adresy a další specifikace, například jaký python má daný host použít. 

\subsubsection{Modules}

Jsou to jednotlivé části kódu, které bude Ansible spouštět. Každý modul má speciální použití. Vše od správy uživatelů ( user ) přes nastavení systému ( systemd ) až k instalovaní balíčků ( apt, yum ). Můžeme spustit jeden modul v tasku, nebo více v playbooku. Pro přehlednost neuvádím všechnz možné moduly, jelikož je jich přes tři tisíce. 

\subsubsection{Tasks}

Jsou jednotky, které se museji provést. Nejčasteji specifikované v deployment souboru. 

\subsubsection{Playbooks}

Je seřazený seznam tasků, které se musí vykonat. Ničemu neuškodí pokud se plazbook spustí znovu, protože Asnible skontroluje stav daného tasku. Playbooky jsou psané podle konvenci YAMLu. 
